\documentclass[a4paper,12pt]{article}
\author{Chifeng Wen}
\title{Literary Review: Simulation of Piano Sound through Physical Modeling}
\begin{document}
\maketitle
\date

\begin{abstract}
        Physical modeling is one of the techniques widely used to reproduce the
        sound of acoustic music instrucments. This technique is 
        superior to most others which require pre-recorded samples. 
        It's effective in the sense that it responses well to the variation of 
        input the music instrument is capable of. 
        In particular, Piano sound synthesis is almost considered case-closed thanks to 
        decades of study with many accurate physical models proposed. 
        In this article, one of the hammer-string models\cite{model} is discussed.
\end{abstract}

\section*{Physical Derivation}

\section*{Solving the Model}

\section*{Results}

\section*{Conclusion}

\begin{thebibliography}{9}
        \bibitem{model}
                J. Bensa, S. Bilbao, R. Kronland-Martinet, and J. O. Smith III
                \textit{"The simulation of piano string vibration: From physical model to 
                finite difference schemes and digital waveguides"}.
                J. Acoust. Soc. Am. 114, 1095–1107 (2003).
\end{thebibliography}

\end{document}
