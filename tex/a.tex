\documentclass[a4paper,12pt]{article}
\usepackage{amsmath}

\author{Chifeng Wen}
\title{Implementation of Piano Sound Synthesis through Physical Modeling}
\begin{document}
\maketitle
\date

\begin{abstract}
        Physical modeling is one of the techniques widely used to reproduce the sound of acoustic music instrucments. This technique is superior to most others which require pre-recorded samples because it can accurately represent the dynamics of the sound coming from the variation of input that the music instrument is capable of. In particular, Piano sound synthesis has considerably improved thanks to a decade study of each phsyical component involved in the generation and propagation of the sound. In this article, I'll discuss and piece up each part of the physical model to form the complete process of sound synthesis as well as the implementation of this process.
\end{abstract}

\section*{Physical Models}
		The simulation of the piano sound involves four coupled systems: the hammer, the string, the soundboard and the room. Initially, the felt wrapped hammer hits on the piano string. The hammer compresses and the impulse is then transferred to the string to form a transverse wave and propagate. The string wave travels to the extremity which connects to the bridge that is fixed on the soundboard on which some of the energy is transferred. At the same time, the wave returns and affects the hammer before it leaves the contact point. The soundboard vibrates and displaces a large area of air which forms a pressure wave that propagates within the room. The displacement also acts back on the bridge which again affects the string. A recording device in the room detects the pressure wave which is what we call -- the sound.
		
\subsection*{Hammer Compression}
		Let's define \[u(t) = h(t) - y(0, t) \] where h is hammer displacement, y is string displacement. This gives the amount of hammer compression which makes things easier to work on later on. Unlike Hooke's spring law, non-linearity is observed\cite{oldmodel} in this type of felt compression coming from piano hammering. In particular, the hammer force is exponential to the amount of compression generated during the contact,
		\[ F(u(t)) = F_0 u^p \]
		In addition, dynamics measurement shows that hysteresis is prominent during the compression. Without taking hysteresis into account, the sound of generated by this hammer is sharper and more metalic, which couldn't faithfully represent the real piano sound. So a term representing the "memory" of the hammer is added\cite{hammer_and_params},
		\[ F(u(t)) = F_0[u^p - \frac{\epsilon}{\tau} 
		                  \int_0^t u_{t'}^p exp(\frac{t' - t}{\tau}) dt'] \]
	 	The constants $F_0$, $\epsilon$ and $\tau$ are qualitative paramters to fit the measured result. However, this integration equation is harder to solve numerically. For the purpose of implementation, the following approximation to this model is much more practical while the difference is hardly noticed\cite{hammer_and_params}. 
		\[ F(u(t)) = Q_0 [u^p + \alpha \frac{d(u^p)}{dt}] \]
		The constants are reported to regress into the following polynomials\cite{hammer_and_params} with n being the nth key on a 88 keys piano, which starts from the note $A_0$
		\begin{multline}
		\begin{aligned}
		&Q_0 = 183 exp(0.045n) \\
		&p = 3.7 + 0.015n \\
		&\alpha = 259.5 - 0.85n + 6.6 \times 10^{-2} n^2 - 
		         1.25 \times 10^{-3} n^3 + 1.172 \times 10^{-5} n^4
		\end{aligned}
		\end{multline}
		The mass of this type of hammer $M_h$ can be modeled as $11.074 - 0.074n + 10^{-4}n^2$.

\subsection*{String Motion}
	The string model originates from the basic wave equation with wave speed $c_s^2$:
	\[ \frac{\partial^2 y}{\partial t^2} = c_s^2 \frac{\partial^2 y}{\partial x^2} \]
	Then we take into account the fact that piano string is stiff where an extra restoring force proportional to the angle of bending kicks in, resulting the phenomenon in which higher frequency components travel slower than those in lower frequency. So we add this dispersiveness to the wave equation and get
	\[ \frac{\partial^2 y}{\partial t^2} = c_s^2 \frac{\partial^2 y}{\partial x^2} + \epsilon \frac{\partial^4 y}{\partial x^4} \]
	where $\epsilon = -r_s^2 \sqrt{\frac{E_s}{\rho_s}}$, $E_s$ is the Young's modulus for the string. $r_s$ and $\rho_s$ are the radius and density of the string, respectively. However, there was attempt\cite{param_fit} that discards the physical meaning of these parameters and uses $c_s$ and $\epsilon$ as fitting facilities. But we don't perform this "optimization" because it tends to cover the actual problem in the model. Later study revealed it was the overlook of the hysteresis of the hammer that introduced the bias.
	
	Now we introduce the damping effect of the string wave due to the loss of energy from air friction and the transfer of energy from the string to the bridge and to the soundboard thereafter. The soundboard continues to lose energy due to air friction and mechanical impedance. For an ideal harmonic oscillator, the decay is proportional to the tranverse velocity 
	\[ 
		\frac{\partial^2 y}{\partial t^2} = 
	   			c_s^2 \frac{\partial^2 y}{\partial x^2} + 
	       		\epsilon \frac{\partial^4 y}{\partial x^4} - 
	   			\alpha_1 \frac{\partial y}{\partial t} 
	\]
	However, in reality, a frequency dependent loss is observed\cite{decay}. The resonance time has the form\cite{string_model}
	\[ \frac{1}{\tau} = \alpha_1 + \alpha_2 \omega^2 \]
	where $\tau$ is the decay rate and $\omega$ is the fundamental frequency. Also, put in the acceleration due to hammer force, we have our final model to be
	\[ \frac{\partial^2 y}{\partial t^2} = 
	   c_s^2 [\frac{\partial^2 y}{\partial x^2} - \epsilon \frac{\partial^4 y}{\partial x^4}] - 
	   \alpha_1 \frac{\partial y}{\partial t} + 
	   \alpha_2 \frac{\partial^3 y}{\partial t^3} + 
	   \frac{1}{M_h} F(u(t))g(x,x_0) \]
	   $M_h$ is the hammer mass. $g(x,x_0)$ describes the distribution of the force during the contact. The actual distribution of the force has not been measured. But, in practice, we can just use a cutoff window since the surface of contact of a piano hammer is a square.
	
\subsection*{Soundboard Motion}
	\[ \]

\subsection*{Propagation through Air}
		\begin{multline}
		\begin{aligned}
			&\rho_a \frac{\partial v_x}{\partial t} = -\frac{\partial p}{\partial x},\\
			&\rho_a \frac{\partial v_y}{\partial t} = -\frac{\partial p}{\partial y},\\
			&\rho_a \frac{\partial v_z}{\partial t} = -\frac{\partial p}{\partial z}	,\\
			&\frac{\partial p}{\partial t} = 
				\rho_a c_a^2 [-\frac{\partial v_x}{\partial x}
				              -\frac{\partial v_y}{\partial y}
				              -\frac{\partial v_z}{\partial z}]
		\end{aligned}
		\end{multline}


\section*{Solving the Models}
		I employ the explicit finite difference method to numerically solve the DE that describes each system.

\subsection*{Hammer Compression}
		\[ F(u(t)) = Q_0 [u^p + \alpha \frac{d(u^p)}{dt}] \]
		can be written as
		\[ M_h \frac{d^2 h_t}{dt^2} - 
		   \alpha Q_0 p u^{p-1} \frac{dh_t}{dt} - 
		   Q_0 u^p = 0 \]
		where $ u(t) = h(t) - y(x_0,t)$. Take the step size $h$, and discretize the equation using central difference,
		\[ M_h \frac{h_{t+1} - 2h_t + h_{t-1}}{h^2} -
		   \alpha Q_0 p u_t^{p-1} \frac{h_{t+1} - h_{t}} {h} - 
		   Q_0 u_t^p = 0 \]
		which can be arranged to the recurrence relation,
		\[ h_{t+1} = (M_h - \alpha h Q_0 p u_t^{p-1})^{-1}
		              [M_h(2h_t - h_{t-1}) -
		               \alpha h Q_0 p u_t^{p - 1} h_t +
		               h Q_0 u_t^p] \] 
		where $$u_t = h_t - y_{x_0,t}$$
		To solve this recurrence relation, we need to know the value of $u_0$ and $u_1$. Given the hammer's initial velocity and knowing its initial contact point, we have the IVP
		\[ \frac{du}{dt} = v_t, u_0 = y(x_0, 0) \]	
		where $x_0$ is the contact point.
		Therefore, we can then find the values of $u_0$ and $u_1$ by solving the system as $t = 1$
		\[
		\begin{cases}
			\begin{array}{lcl}
			\begin{aligned}
				h_2 = &[M_h - \alpha h Q_0 p (h_1 - y_{x_0,1})^{p-1}]^{-1} \\
		              & [h Q_0 (h_1 - y_{x_0,1})^p  -
		                 \alpha h Q_0 p (h_1 - y_{x_0,1})^{p - 1} h_t +
		                 M_h(2h_1 - h_0)]
		    \end{aligned}
		    \end{array} \\
				h_1 = h_0 + hv_h \\
				h_0 = y_{x_0,0}
		\end{cases}
		\]
		
\subsection*{String Motion}
		\[ \frac{\partial^2 y}{\partial t^2} = 
	   c_s^2 [\frac{\partial^2 y}{\partial x^2} - \epsilon \frac{\partial^4 y}{\partial x^4}] - 
	   \alpha_1 \frac{\partial y}{\partial t} + 
	   \alpha_2 \frac{\partial^3 y}{\partial t^3} + 
	   \frac{1}{M_h} F(u(t)) g(x,x_0) \]
	   One can argue that $y_{x,t}$ is useless if the initial contact between the hammer and string is short enough and the string can no longer affect the hammer hence we can just solve to the second order. But study shows the contact time is much longer than what one may think. I wish to faithfully describe the actually physical event rather than simplifying early on then using the measured results to fit the paramters later.
	   
	   The discretization goes as follow taking $h_t$ and $h_x$ be the time and spatial step size, respectively (The acceleration due to hammering is abbreviated from the following calculations).
	   \begin{multline}
	   \frac{y_{x,t+1} - 2y_{x,t} + y_{x,t-1}}{h_t^2} =\\
	      c_s^2[\frac{y_{x+1,t} - 2y_{x,t} + y_{x-1,t}}{h_x^2} -
	            \epsilon \frac{y_{x+2,t} - 4y_{x+1,t} + 6y_{x,t} - 4y_{x-1,t} + y_{x-2,t}s}{h_x^4}] - \\
	            \alpha_1 \frac{y_{x,t+1} - y_{x,t}}{h_t} + 
	            \alpha_2 \frac{y_{x,t+1} - 3y_{x,t} + 3y_{x,t-1} - y_{x,t-2}}{h_t^3} 
		\end{multline}
		Our goal is to find out the transverse velocity at the bridge $\frac{\partial y}{\partial t}(L, t)$ and the displacement at the contact point $y(x_0, t)$.
		\begin{multline}
	    y_{x,t+1} =\\
	      (h_t + \alpha_1 h_t^2 - \alpha_2)^{-1}
	      \{2h_t y_{x,t} - h_t y_{x,t-1} +\\
	      h_t^3 c_s^2[\frac{y_{x+1,t} - 2y_{x,t} + y_{x-1,t}}{h_x^2} -
	            \epsilon \frac{y_{x+2,t} - 4y_{x+1,t} + 6y_{x,t} - 4y_{x-1,t} + y_{x-2,t}}{h_x^4}] + \\
	            \alpha_1 h_t^2 y_{x,t} + 
	            \alpha_2 (y_{x,t+1} - 3y_{x,t} + 3y_{x,t-1} - y_{x,t-2})\}
		\end{multline}
		Now, we add in the acceleration due to hammer force and rearrange the equation,
		\begin{multline}
		\begin{aligned}
	    y_{x,t+1} =&\\
	      &(h_t + \alpha_1 h_t^2 - \alpha_2)^{-1}\\
	      &[(2h_t - 
	        \frac{2h_t^3c_s^2}{h_x^2} - 
	        \frac{6h_t^3c_s^2\epsilon}{h_x^4} + 
	        \alpha_1h_t^2 - 3\alpha_2)y_{x,t} +\\
	      &(\frac{h_t^3c_s^2}{h_x^2} + 
	         \frac{4h_t^3c_s^2\epsilon}{h_x^4})y_{x+1,t} +
	        (\frac{h_t^3c_s^2}{h_x^2} + 
	         \frac{4h_t^3c_s^2\epsilon}{h_x^4})y_{x-1,t} + \\
	       &(3\alpha_2 - h_t)y_{x,t-1} - \alpha_2 y_{x,t-2} + 
	        \frac{1}{M_h} h_t^3 f(u_t) g(x,x_0) ]
	    \end{aligned}
		\end{multline}
		As $t = 2, x = x_0$, there are 6 unknowns to attack to bootstrap the evolution of this system. I first consider the fixed extremities conditions. The assumption that the extremities are fixed is not true in the physical piano. The condition only serves as an approximation. The details of how the extremities move as the wave arrives require further study to understand and better modeled. Also, the string is assumed to be at rest at $t = 0$. The boundary and initial conditions go as follow, 
		\[ y(0,t) = y(L,t) = 0 \]
		\[ y(x,0) = 0 \]
		and
		\[ \frac{\partial^2 y}{\partial x^2} (0,t) = 
		   \frac{\partial^2 y}{\partial x^2} (L,t) = 0 \]
		($L$ is the string length).
		
		In general, this DE cannot be solved explicitly with these boundary conditions, but we can play a trick here to approximate the initial event to allow us to run the recurrence relation. Initially, at time $t = 0$, the string is at rest. Therefore, $y_{x_0,0} = 0$. At time $t = 1$, the problem arises as we don't have the information of $y_{x_0,-2}$ and $y_{x_0,-1}$ to begin with. We'll resolve this problem by ignoring the string stiffness and damping effect for the first 3 time steps and treat it as a simple wave equation plus the hammer acceleration. After that, we swap the precise model back in and continue the evolution using the displacements of the first 3 time steps.

\subsection*{Soundboard Motion}


\subsection*{Propagation through Air}

\section*{Stability and Error Analysis}
		
\section*{Results}

\section*{Conclusion}

\begin{thebibliography}{9}
        \bibitem{oldmodel}
                J. Bensa, S. Bilbao, R. Kronland-Martinet, and J. O. Smith III
                \textit{"The simulation of piano string vibration: From physical model to 
                finite difference schemes and digital waveguides"}.
                The Journal of the Acoustical Society of America. 114, 1095–1107 (2003).
		\bibitem{decay}
			Thomas Rossing
			\textit{The Science of String Instruments}.
            Springer Science \& Business Media. 386-387 (2010).

		\bibitem{param_fit}
			Julien Bensa, Olivier Gipouloux, Richard Kronland-Martinet, and NHF
			\textit{"Parameter fitting for piano sound synthesis by physical modeling"}
			The Journal of the Acoustical Society of America. 118, 495 (2005).
		            
        \bibitem{string_model}
        	Antoine Chaigne and Anders Askenfelt
        	\textit{"Numerical simulations of piano strings. I. A physical model
for a struck string using finite difference methods"}
			The Journal of the Acoustical Society of America. 95, 1112 (1994).
		
		\bibitem{hammer_and_params}
			\textit{"Physical modelling of the piano string scale"}
			Applied Acoustics. Volume 69, Issue 11, 977–984 (2008).
		
		\bibitem{grid}
			D. Botteldooren
			\textit{"Acoustical finite‐difference time‐domain simulation in a quasi‐Cartesian grid"}
			The Journal of the Acoustical Society of America 95, 2313 (1994).
		
		\bibitem{room}
			D. Botteldooren
			\textit{"Finite-difference time-domain simulation of low-frequency room acoustic problems"}
			The Journal of the Acoustical Society of America 98, 3302 (1995).
\end{thebibliography}
\end{document}
